%% Generated by Sphinx.
\def\sphinxdocclass{report}
\documentclass[letterpaper,11pt,english]{sphinxmanual}
\ifdefined\pdfpxdimen
   \let\sphinxpxdimen\pdfpxdimen\else\newdimen\sphinxpxdimen
\fi \sphinxpxdimen=.75bp\relax

\PassOptionsToPackage{warn}{textcomp}
\usepackage[utf8]{inputenc}
\ifdefined\DeclareUnicodeCharacter
% support both utf8 and utf8x syntaxes
  \ifdefined\DeclareUnicodeCharacterAsOptional
    \def\sphinxDUC#1{\DeclareUnicodeCharacter{"#1}}
  \else
    \let\sphinxDUC\DeclareUnicodeCharacter
  \fi
  \sphinxDUC{00A0}{\nobreakspace}
  \sphinxDUC{2500}{\sphinxunichar{2500}}
  \sphinxDUC{2502}{\sphinxunichar{2502}}
  \sphinxDUC{2514}{\sphinxunichar{2514}}
  \sphinxDUC{251C}{\sphinxunichar{251C}}
  \sphinxDUC{2572}{\textbackslash}
\fi
\usepackage{cmap}
\usepackage[T1]{fontenc}
\usepackage{amsmath,amssymb,amstext}
\usepackage{babel}



\usepackage{times}
\expandafter\ifx\csname T@LGR\endcsname\relax
\else
% LGR was declared as font encoding
  \substitutefont{LGR}{\rmdefault}{cmr}
  \substitutefont{LGR}{\sfdefault}{cmss}
  \substitutefont{LGR}{\ttdefault}{cmtt}
\fi
\expandafter\ifx\csname T@X2\endcsname\relax
  \expandafter\ifx\csname T@T2A\endcsname\relax
  \else
  % T2A was declared as font encoding
    \substitutefont{T2A}{\rmdefault}{cmr}
    \substitutefont{T2A}{\sfdefault}{cmss}
    \substitutefont{T2A}{\ttdefault}{cmtt}
  \fi
\else
% X2 was declared as font encoding
  \substitutefont{X2}{\rmdefault}{cmr}
  \substitutefont{X2}{\sfdefault}{cmss}
  \substitutefont{X2}{\ttdefault}{cmtt}
\fi


\usepackage[Bjarne]{fncychap}
\usepackage{sphinx}

\fvset{fontsize=\small}
\usepackage{geometry}


% Include hyperref last.
\usepackage{hyperref}
% Fix anchor placement for figures with captions.
\usepackage{hypcap}% it must be loaded after hyperref.
% Set up styles of URL: it should be placed after hyperref.
\urlstyle{same}

\usepackage{sphinxmessages}
\setcounter{tocdepth}{1}


        \usepackage{charter}
        \usepackage[defaultsans]{lato}
        \usepackage{inconsolata}
    

\title{FFE User\textquotesingle{}s Guide}
\date{May 17, 2020}
\release{}
\author{TinkerTools Organization}
\newcommand{\sphinxlogo}{\vbox{}}
\renewcommand{\releasename}{}
\makeindex
\begin{document}

\pagestyle{empty}
\sphinxmaketitle
\pagestyle{plain}
\sphinxtableofcontents
\pagestyle{normal}
\phantomsection\label{\detokenize{index::doc}}



\chapter{Introduction}
\label{\detokenize{text/introduction:introduction}}\label{\detokenize{text/introduction::doc}}

\section{What is Force Field Explorer?}
\label{\detokenize{text/introduction:what-is-force-field-explorer}}
Force Field Explorer (FFE) is a graphical user interface to the Tinker suite of molecular modeling tools. The goal is to create a robust environment for computational chemistry and molecular engineering applications. Current features include:
\begin{itemize}
\item {} 
Setup and real\sphinxhyphen{}time visualization of modeling routines, including local/global optimizations, sampling methods, superposition, etc.

\item {} 
Basic 3D molecular visualization (wireframe, tube, spacefilling, ball \& stick, etc.) with special features for force field specific parameters such as Lennard\sphinxhyphen{}Jones radii and partial charge magnitudes.

\item {} 
A comprehensive editor for Tinker keyword files, which contain parameters that control modeling algorithms.

\item {} 
Tinker trajectory playback.

\end{itemize}


\section{Current Release}
\label{\detokenize{text/introduction:current-release}}
The current release of Force Field Explorer is denoted as Version 8.8 and dated July 2020 to mirror the corresponding Tinker version. Dependencies, listed below, are now included in self\sphinxhyphen{}extracting installers for Linux, MacOS and Microsoft Windows.
\begin{itemize}
\item {} 
Tinker version 8.8

\item {} 
Java Runtime Environment (JRE) 1.8 (Java 8) or later

\item {} 
Java3D version 1.6

\end{itemize}

Major improvements over the previous releases of FFE include:
\begin{itemize}
\item {} 
Self\sphinxhyphen{}Extracting Installers for Linux, MacOS and Microsoft Windows. These distributions include Tinker binaries, source code, documentation, and example files. Private JRE and Java3D libraries are bundled with all versions of FFE.

\item {} 
An intuitive, consistent selection mechanism.

\item {} 
Improved keyword and modeling panels.

\item {} 
Support for Tinker internal coordinate, restart and induced dipole file formats, plus loading of Protein Databank files directly from the PDB or from disk. Conversion of PDB files to Tinker format (with automatic atom typing for many biomolecules) is also supported.

\end{itemize}


\section{Future Plans}
\label{\detokenize{text/introduction:future-plans}}
The most important Force Field Explorer development goal is to add further molecular editing and visualization features. Currently, Java3D is used as the graphics toolkit underlying FFE. Java3D was originally developed by Sun, but is now maintained by an open source consortium and is freely available. Other goals include support for other file types and special capabilities for molecular complexes, docking and binding.


\chapter{Installation}
\label{\detokenize{text/installation:installation}}\label{\detokenize{text/installation::doc}}
Force Field Explorer and Tinker are distributed via self\sphinxhyphen{}extracting installers for Linux, Apple MacOS and Microsoft Windows. Included within the installers are binary Tinker executables, source code, example/test files, documentation and the Java environment needed by Force Field Explorer. FFE contains a private JRE version that will not affect any other JREs installed system wide.

For each platform, a launch script/exe located in the top level installation directory named “ffe” will launch Force Field Explorer.

On some platforms, double clicking the ffe.jar file in the “tinker/jar” directory will also launch Force Field Explorer. In this case, your system JRE will be invoked instead of the private JRE bundled with the installation kit. This should behave correctly, as long as your system JRE is a 1.8 version (Java 8) and includes Java3D extensions.

Specific installation instructions for operating systems are provided below:
\begin{itemize}
\item {} 
Linux
\begin{quote}

After downloading the installer, execute it from a shell window. A script to run Force Field Explorer is located in the top\sphinxhyphen{}level installation directory, which defaults to “\$HOME/Tinker\sphinxhyphen{}FFE”. In this case, Tinker binaries would be located in the “\$HOME/Tinker\sphinxhyphen{}FFE/tinker/bin” directory.
\end{quote}

\item {} 
Apple MacOS
\begin{quote}

After downloading the installer, double click to start installation. Force Field Explorer can then be launched by double clicking on the Desktop icon.
\end{quote}

\item {} 
Microsoft Windows
\begin{quote}

After downloading the installer, double click to start installation. Force Field Explorer can then be launched from an icon or the Start Menu.
\end{quote}

\end{itemize}


\chapter{Description of FFE Components}
\label{\detokenize{text/components:description-of-ffe-components}}\label{\detokenize{text/components::doc}}

\section{Molecular Tree}
\label{\detokenize{text/components:molecular-tree}}
The Molecular Tree is a structural hierarchy of each system, used for navigating and making selections. When Force Field Explorer opens a structure file, it attempts to groups atoms into protein/nucleic acid macromolecules and their constituent residues. It also groups ions, water and hetero molecules together. This is done even in the absence of a Tinker sequence file. When a display or color command is chosen, all currently selected groups of atoms are affected.

\noindent\sphinxincludegraphics[width=625\sphinxpxdimen]{{figure1}.png}

Figure 1. The first residue of the enkephalin peptide is selected in the Tree view and is highlighted in blue on the Graphics Window.


\section{3D Graphics and Global Axis}
\label{\detokenize{text/components:d-graphics-and-global-axis}}
Left\sphinxhyphen{}clicking the Global Axis, then dragging performs a rotation about the origin of the entire scene. This is useful for manual docking of two systems. For example, after positioning each system individually in the X\sphinxhyphen{}Y plane by dragging them with the mouse, a rotation about the Y\sphinxhyphen{}axis allows orientation in the Y\sphinxhyphen{}Z plane. Right\sphinxhyphen{}clicking the Global Axis, then dragging performs a translation of the entire scene.

\noindent\sphinxincludegraphics[width=625\sphinxpxdimen]{{figure2}.png}

Figure 2. Dragging the Graphics Axis causes the entire scene to rotate about the global origin. The scene above is rotated 180 degrees around the Y\sphinxhyphen{}axis.


\section{Keyword Editor}
\label{\detokenize{text/components:keyword-editor}}
The Keyword Editor allows modification of keyword files that control various aspects of Tinker calculations. If a Modeling Command is executed on a system, the corresponding Keyword File is automatically saved. Modifications can also be saved at any time using one of the save buttons. Any text or keywords that Force Field Explorer does not recognize are considered “Comments” and are appended to the end of saved key files. As an example, the enkephalin keyword control file is shown below.

\noindent\sphinxincludegraphics[width=625\sphinxpxdimen]{{figure3}.png}

Figure 3. The Keyword Editor Panel displaying the enkephalin keyword file.


\section{Modeling Commands}
\label{\detokenize{text/components:modeling-commands}}
The Modeling Commands panel of Force Field Explorer allows launching of most of the Tinker programs. After selecting a routine and configuring its modifying arguments, selecting the “play” button starts the job running. If Tinker will modify the coordinates of the system, Force Field Explorer will automatically connect to the routine and show progress interactively. If the routine completes before Force Field Explorer is exited, the textual results in the log file are loaded into the Logs panel. If Force Field Explorer exits while one or more routines are running, they continue in the background unless explicitly killed by the user. For example, the “Optimize” command has been selected for enkephalin (Figure 4).

\noindent\sphinxincludegraphics[width=625\sphinxpxdimen]{{figure4}.png}

Figure 4. The Tinker OPTIMIZE program selected in the Modeling Commands panel and will execute on the active system, in this case the enkephalin peptide.


\section{Activity Logs}
\label{\detokenize{text/components:activity-logs}}
The Logs panel is a simple text editor where output logged from Tinker routines is automatically loaded. It can also be used to edit any text file, for example Tinker coordinate or keyword files.

\noindent\sphinxincludegraphics[width=625\sphinxpxdimen]{{figure5}.png}

Figure 5. The Log panel showing output logged from running Tinker OPTIMIZE on the enkephalin peptide.

If a modeling command changes the coordinates of an open structure, the final structure generated is re\sphinxhyphen{}loaded onto the open structure used to initiate the command.

\noindent\sphinxincludegraphics[width=625\sphinxpxdimen]{{figure6}.png}

Figure 6.  The coordinates for the final local minimum of enkephalin (enkephalin.xyz\_2) are loaded onto the originally opened structure, enkephalin.xyz, used as input to the Tinker OPTIMIZE command.


\chapter{Description of FFE Menus}
\label{\detokenize{text/menus:description-of-ffe-menus}}\label{\detokenize{text/menus::doc}}

\section{File Menu}
\label{\detokenize{text/menus:file-menu}}
Force Field Explorer currently reads Tinker coordinate files (\sphinxstyleemphasis{.xyz, *.int), restart files (}.dyn), archive files (\sphinxstyleemphasis{.arc), induced dipole files (*u) and Protein Databank files (}.pdb). However, file writing has only been implemented for Tinker Cartesian coordinate files.
\begin{itemize}
\item {} 
Open

\end{itemize}

An attempt is made to open the selected file. Currently, Tinker coordinate files, Archive files, and Protein Databank files are recognized.
\begin{itemize}
\item {} 
Save As

\end{itemize}

Saves the active file under a different name. (This command is currently only supported for Cartesian coordinate Tinker files)
\begin{itemize}
\item {} 
Close

\end{itemize}

Close the active file.
\begin{itemize}
\item {} 
Close All

\end{itemize}

Close all open files.
\begin{itemize}
\item {} 
Download from PubChem

\end{itemize}

Entering a valid molecule name will download the corresponding structure from the NIH PubChem database, convert atom types to the “Tiny” force field, and save the Tinker .xyz file to the NIH Downloads directory, then open it.
\begin{itemize}
\item {} 
Download from NCI

\end{itemize}

Entering a valid molecule name will download the corresponding structure from the NCI Cactus database, convert atom types to the “Tiny” force field, and save the Tinker .xyz file to the NIH Downloads directory, then open it.
\begin{itemize}
\item {} 
Download from PDB

\end{itemize}

Entering a valid 4 character accession will download the corresponding PDB file, save it to the current Force Field Explorer working directory, then open it.
\begin{itemize}
\item {} 
Load Restart Data

\end{itemize}

Tinker Restart files ({\color{red}\bfseries{}*}.dyn) contain atomic positions, velocities and two sets of accelerations (current and previous step), but no connectivity information. Loading a Restart file on top of an open Tinker coordinate file allows visualization of velocities and accelerations.
\begin{itemize}
\item {} 
Load Induced Data

\end{itemize}

Tinker Induced Dipole files ({\color{red}\bfseries{}*}.*u) contain induced dipole vectors. They are produced during a molecular dynamics routine using the Tinker “save\sphinxhyphen{}induced” keyword. Loading an Induced Dipole file on top of an open Tinker coordinate file allows visualization of induced dipoles.
\begin{itemize}
\item {} 
Exit

\end{itemize}

Exit shuts Force Field Explorer down, saving user preferences.


\section{Selection Menu}
\label{\detokenize{text/menus:selection-menu}}
The set of nodes currently selected are highlighted in the Molecular Tree. If the shift key his pressed while clicking on two nodes, all nodes in between are selected. Alternatively, by holding down the control key while clicking on a node, either in the Tree or Graphics window, the selection state of a node can be toggled. To use Graphics based picking, the Graphics Picking checkbox under the Picking menu must be checked.
\begin{itemize}
\item {} 
Select All

\end{itemize}

All open systems are selected.
\begin{itemize}
\item {} 
Restrict to Selections

\end{itemize}

Nodes that are not selected are made invisible.
\begin{itemize}
\item {} 
Highlight Selections

\end{itemize}

Selected nodes are highlighted in the Graphics window.
\begin{itemize}
\item {} 
Set Selection Color

\end{itemize}

Choose the color for selected atoms.
\begin{itemize}
\item {} 
Label Selected Atoms

\end{itemize}

Selected atoms are labeled by number and description.
\begin{itemize}
\item {} 
Label Selected Residues

\end{itemize}

Selected residues are labeled by their respective three letter identifiers.
\begin{itemize}
\item {} 
Set Label Font Size

\end{itemize}

The default label size is 12 point.
\begin{itemize}
\item {} 
Set Label Font Color

\end{itemize}

The default font color is white.


\section{Display Menu}
\label{\detokenize{text/menus:display-menu}}
We have intentionally maintained a simple, but elegant rendering mechanism. The Force Field Explorer data structure is self\sphinxhyphen{}rendering, meaning that each object “knows how to draw itself”. Rendering commands are “Top\sphinxhyphen{}Down”, meaning that commands filter down through the molecular tree, with each node responding and/or passing commands to its children. This paradigm positions Force Field Explorer to easily scale up to multi\sphinxhyphen{}level molecular engineering/nanotechnology modeling projects. What we want to avoid is a rendering/data architecture built upon “an array of atoms”. In our opinion, “an array of atoms” infrastructure can not achieve our molecular engineering design goals. While a graphics change is being rendered, one additional graphics command will be cached. Other commands are ignored while the 3D view catches up. Changes are applied to all selected nodes in the Molecular Tree View.
\begin{itemize}
\item {} 
Wireframe

\item {} \begin{description}
\item[{Tube}] \leavevmode
Slow for large systems (more than a few thousand atoms).

\end{description}

\item {} 
Spacefill

\item {} 
Ball \& Stick

\end{itemize}

Slow for large systems (more than a few thousand atoms).
\begin{itemize}
\item {} 
Invisible

\item {} 
RMIN

\end{itemize}

Similar to Spacefill, but the sphere radii are set to the Lennard\sphinxhyphen{}Jones minima for the given atom.
\begin{itemize}
\item {} 
Fill

\end{itemize}

Fill rendered polygons.
\begin{itemize}
\item {} 
Points

\end{itemize}

Render only vertices.
\begin{itemize}
\item {} 
Lines

\end{itemize}

Render edges between vertices.
\begin{itemize}
\item {} 
Preferences \textendash{} Radius

\end{itemize}

A scale factor applied to all rendered spheres and cylinders.
\begin{itemize}
\item {} 
Preferences \textendash{} Wireframe Thickness

\end{itemize}

This specifies the thickness of wireframe representations in pixels.
\begin{itemize}
\item {} 
Preferences \textendash{} Detail

\end{itemize}

Changes the number of vertices used in creating spheres and cylinders. More vertices (a higher number on the slider) create a smoother appearance, but slower rotations.


\section{Color Menu}
\label{\detokenize{text/menus:color-menu}}\begin{itemize}
\item {} 
Monochrome

\end{itemize}

Black \& White rendering.
\begin{itemize}
\item {} 
CPK

\end{itemize}

Atom based colors.
\begin{itemize}
\item {} 
Residue

\end{itemize}

Color based on amino/nucleic acid.
\begin{itemize}
\item {} 
Structure

\end{itemize}

Color based on secondary structure specified in PDB files.
\begin{itemize}
\item {} 
Polymer

\end{itemize}

A different color is assigned to each polymer.
\begin{itemize}
\item {} 
Partial Charge

\end{itemize}

Coloring is based on partial charges that correspond to atom types specified in Tinker coordinate files.
\begin{itemize}
\item {} 
Vector Magnitude

\end{itemize}

Coloring is based on vector magnitude, where the vector may represent velocity, acceleration, induced dipole or force depending on user choice.
\begin{itemize}
\item {} 
User Color

\end{itemize}

Arbitrary colors can be assigned to selected atoms by choosing the “Apply User Color to Selections” menu item. The “User Color” can then be selected to invoke this scheme.
\begin{itemize}
\item {} 
Apply User Color to Selections

\end{itemize}

See Above.
\begin{itemize}
\item {} 
Set User Color

\end{itemize}

Specify the color to use when selecting the “Apply User Color to Selections” option.


\section{Options Menu}
\label{\detokenize{text/menus:options-menu}}\begin{itemize}
\item {} 
Active System

\end{itemize}

Coordinate controls for the active system.
* Reset Rotation
Removes any rotations applied to the active system.
* Reset Translation
Removes any translation applied to the active system.
* Reset Rotation and Translation
Removes any rotation/translation applied to the active system.
* Rotate About Center of Mass
Specifies the active system should be rotated about its center of mass.
* Reset About Pick
Specifies the active system should be rotated about the node selected by graphics picking.
\begin{itemize}
\item {} 
Drag

\end{itemize}

Specify which system is controlled by mouse clicking/dragging within the Graphics window.
* Active System
Regardless of mouse position, the Active system is controlled.
* System Below Mouse
To move a system, it must be picked by clicking over it with the mouse.
\begin{itemize}
\item {} 
Reset Global Zoom

\end{itemize}

Reset the global zoom that has been applied to the entire scene.
\begin{itemize}
\item {} 
Reset Global Translation

\end{itemize}

Remove any global translation that has been applied to the entire scene.
\begin{itemize}
\item {} 
Reset Global Rotation

\end{itemize}

Remove any global rotation that has been applied to the entire scene.
\begin{itemize}
\item {} 
Reset Global View

\end{itemize}

Remove any global rotation and translation, then reset the global zoom.
\begin{itemize}
\item {} 
Toggle Hydrogens

\end{itemize}

Show/Hide hydrogen atoms.
\begin{itemize}
\item {} 
Full Screen

\end{itemize}

Make the graphics window fill the entire screen.
\begin{itemize}
\item {} 
Set Background Color

\end{itemize}

Change the color of the graphics window background. This is useful for switching to a white background before capturing images that might be printed.


\section{Picking Menu}
\label{\detokenize{text/menus:picking-menu}}
Graphics picking can be used to select nodes anywhere in the structural hierarchy. By holding the control key while picking, the selection state of the picked node is toggled between selected/not selected. However, holding the shift button down while picking currently has no effect.
\begin{itemize}
\item {} 
Graphics Picking

\end{itemize}

Selecting this checkbox turns on Graphics Picking.
* Atom
Individual atoms will be picked.
* Bond
A Bond will be picked, with consecutive clicks cycling through all Bonds made by the picked atom.
* Angle
An Angle will be picked, with consecutive clicks cycling through all Angles made by the picked atom.
* Dihedral
A Dihedral will be picked, with consecutive clicks cycling through all Dihedrals made by the picked atom.
* Residue
* Polymer
* Molecule
* System
* Measure Distance
After selecting two atoms the distance between them will be reported in the Force Field Explorer console.
* Measure Angle
After selecting three atoms the angle formed between them will be reported in the Force Field Explorer console.
* Measure Dihedral
After selecting four atoms the dihedral formed between them will be reported in the Force Field Explorer console.
\begin{itemize}
\item {} 
Set Graphics Picking Color

\end{itemize}

Picked nodes will be highlighted in the color selected using this control.


\section{Trajectory Menu}
\label{\detokenize{text/menus:trajectory-menu}}
Tinker trajectories are stored in Archive files (\sphinxstyleemphasis{.arc), which are simply a concatenation of coordinate files (}.xyz, {\color{red}\bfseries{}*}.int).
\begin{itemize}
\item {} 
Oscillate

\end{itemize}

When the end of a trajectory is reached, it is played in the reverse direction instead of restarting at the beginning.
\begin{itemize}
\item {} 
Frame

\end{itemize}

Advance to an arbitrary frame of the active trajectory.
\begin{itemize}
\item {} 
Speed

\end{itemize}

Choose a frame rate for trajectory playback. If 20 frames per second is entered, for example, a delay of 50 msec between coordinate changes is enforced. Therefore, the actual frame rate will be slower than what is specified if rendering for the system is slow.
\begin{itemize}
\item {} 
Skip

\end{itemize}

Instead of displaying every frame during trajectory playback, only every nth frame will be rendered.
\begin{itemize}
\item {} 
Play

\end{itemize}

Play the active trajectory.
\begin{itemize}
\item {} \begin{description}
\item[{Stop}] \leavevmode
Stop the active trajectory.

\end{description}

\item {} \begin{description}
\item[{Step Forward}] \leavevmode
Advance the active trajectory one frame.

\end{description}

\item {} \begin{description}
\item[{Step Back}] \leavevmode
Rewind the active trajectory one frame.

\end{description}

\item {} \begin{description}
\item[{Reset}] \leavevmode
Return the active trajectory to the first frame.

\end{description}

\end{itemize}


\section{Simulation Menu}
\label{\detokenize{text/menus:simulation-menu}}
Tinker sampling and optimization methods are configured to start a server that listens for Force Field Explorer clients. Once a client \textendash{} server connection is made, simulation information can be sent between Tinker and Force Field Explorer. Multiple Force Field Explorer clients can connect to the same Tinker simulation.
\begin{itemize}
\item {} 
Release Job

\end{itemize}

If Force Field Explorer is connected to a Tinker simulation, the connection will be closed.
\begin{itemize}
\item {} 
Connect to Local Job

\end{itemize}

Force Field Explorer will attempt to connect to a Tinker simulation executing on the current computer.
\begin{itemize}
\item {} 
Connect to Remote Job

\end{itemize}

Force Field Explorer will attempt to connect to a Tinker simulation executing on a remote computer.
\begin{itemize}
\item {} 
Set Remote Job Address

\end{itemize}

Specify the IP address of the remote computer where a Tinker simulation is executing. No domain name server is used, so the address should be of the form: XXX.XXX.XXX.XXX


\section{Export Menu}
\label{\detokenize{text/menus:export-menu}}
Currently, Force Field Explorer can save screenshots uses JPEG and PNG formats. More formats will be supported when we target Java 1.5 (this release targets Java 1.4.2). In most cases, PNG gives better image quality when using Force Field Explorer than JPEG, probably because the default JPEG configuration we are using chooses small image size over preservation of detail.
\begin{itemize}
\item {} 
Capture Graphics

\end{itemize}

Capture the Graphics window to an image file.
\begin{itemize}
\item {} \begin{description}
\item[{PNG}] \leavevmode
Use the PNG image format.

\end{description}

\item {} \begin{description}
\item[{JPEG}] \leavevmode
Use the JPEG image format.

\end{description}

\end{itemize}

\textgreater{} Window
Force Field Explorer components can be customized to some extent to suit user preferences. Many user configurable choices are saved between sessions.
\begin{itemize}
\item {} 
Reset Panes

\end{itemize}

This resets the slider between the Molecular Tree and Graphics Windows to the default position. The Graphics window is a heavy weight component (so hardware graphics acceleration can be used), while most Force Field Explorer components are light weight “Swing” components. Unfortunately, this effect prevents the user from dragging the slider and shrinking the Graphics window.
\begin{itemize}
\item {} \begin{description}
\item[{Show Tree}] \leavevmode
Show/Hide the Tree view.

\end{description}

\item {} \begin{description}
\item[{Show Tool Bar}] \leavevmode
Show/Hide the Tool Bar.

\end{description}

\item {} 
Show Global Axes

\end{itemize}

Show/Hide the Global Axes. The Global Axes are used both as a visual cue and to affect global rotations and translations.
\begin{itemize}
\item {} \begin{description}
\item[{Show Console}] \leavevmode
Show the command line Console.

\end{description}

\item {} \begin{description}
\item[{Java Metal Look \& Feel}] \leavevmode
This is the default appearance and should be consistent across platforms.

\end{description}

\item {} 
Windows/Macintosh/Motif Look \& Feel

\end{itemize}

This appearance should cause Force Field Explorer to more closely resemble the native platform.


\section{Help Menu}
\label{\detokenize{text/menus:help-menu}}\begin{itemize}
\item {} 
About

\end{itemize}

This displays the Tinker logo and contact information.


\chapter{Acknowledgments}
\label{\detokenize{text/acknowledgements:acknowledgments}}\label{\detokenize{text/acknowledgements::doc}}
The original version of Force Field Explorer was written by Michael Schnieders, before and during his graduate school stay in Jay Ponder’s laboratory at Washington University in St. Louis. The code is now being developed and maintained by the Ponder group.

A significant contribution to Force Field Explorer was made by Zhiguang “Frank” Gao over the summer of 2002. The Force Field Explorer logo shows an optimized AMOEBA gas phase water dimer, and was provided by Pengyu Ren. The User’s Guide cover showing the crambin protein with 13\sphinxhyphen{}PHE highlighted was produced by Jay Ponder. Special thanks to Alan Grossfield for offering many helpful suggestions. We would also like to thank Eric Reiss for making available his modified version of some Java3D Behaviors, and Pat Niemeyer for his work in developing the BeanShell scripting framework. Important code modernization and updating of FFE was done over the period from 2013 to 2016 by Jeffrey Bigg, Tyler Ponder and Brendan McMurrow. Modernization and unification of the Linux, Windows and MacOS code bases with done during 2017\sphinxhyphen{}18 by Jay Ponder.


\section{Reporting Bugs}
\label{\detokenize{text/acknowledgements:reporting-bugs}}
We apologize in advance for any bugs that remain in Force Field Explorer. Should you happen to perform some command that causes a Java stack trace to be dumped to the console, please copy the output and forward to \sphinxhref{mailto:ponder@dasher.wustl.edu}{ponder@dasher.wustl.edu}. Poor video card drivers can sometimes result in unpredictable Force Field Explorer behavior. If this should happen, try to find the latest drivers for your platform/video card combination.



\renewcommand{\indexname}{Index}
\printindex
\end{document}